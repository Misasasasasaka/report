\documentclass[a4paper,12pt]{ctexart}

\usepackage{geometry}
\geometry{left=2.5cm,right=2.5cm,top=3cm,bottom=3cm}
\usepackage{graphicx}
\usepackage{booktabs}
\usepackage{longtable}
\usepackage{listings}
\usepackage{xcolor}
\usepackage{amsmath, amssymb}
\usepackage{hyperref}
\usepackage[normalem]{ulem}
\usepackage{listings}

\newcommand{\uText}[2][3cm]{\uline{\makebox[#1][c]{#2}}}

\hypersetup{
    colorlinks=true,
    linkcolor=blue,
    citecolor=blue,
    urlcolor=blue
}


\lstset{
    basicstyle=\ttfamily\small,
    numbers=left,
    numberstyle=\tiny,
    keywordstyle=\color{blue},
    commentstyle=\color{gray},
    stringstyle=\color{red},
    frame=single,
    breaklines=true,
    showstringspaces=false
}

\begin{document}

\begin{titlepage}
    \centering
    \vspace*{3cm}
    {\Huge\bf 系统开发工具基础实验报告\\[1.5cm]}
    {\Large\it 实验内容:\uText[4cm]{实验二}\\[0.5cm]}
    {\Large\it 姓名:\uText[4cm]{张家宜} \quad 学号:\uText[5cm]{20240013045}\\[0.5cm]}
    {\Large\it 日期:\today\\[1.5cm]}
    \vfill
    \normalsize
    \vspace*{1cm}
\end{titlepage}

% 目录
\tableofcontents
\newpage


    
\section{练习内容}

\subsection{Shell 工具与脚本}


\subsubsection{命令与 PATH}
\begin{lstlisting}
echo $PATH
which ls
\end{lstlisting}
\texttt{\$PATH} 决定可执行文件的搜索路径 \\ \texttt{which}/\texttt{command -v} 可查看命令解析结果

\subsubsection{目录与导航}
\begin{lstlisting}
pwd
cd ..
ls -lah
\end{lstlisting}
查看当前目录、切换到上级、以更可读方式列目录(含隐藏文件)

\subsubsection{重定向与追加}
\begin{lstlisting}
echo "hello" > out.txt
echo "world" >> out.txt
cat < out.txt
\end{lstlisting}
\texttt{>} 覆盖写入,\texttt{>>} 追加写入,\texttt{<} 将文件作为标准输入

\subsubsection{管道组合}
\begin{lstlisting}
ls / | head -n 5
\end{lstlisting}
将一个命令的输出作为下一个命令的输入(\texttt{|})

\subsubsection{通配(globs)}
\begin{lstlisting}
echo *.txt
echo data/??.csv
\end{lstlisting}
\texttt{*} 匹配任意串,\texttt{?} 匹配单个字符,\texttt{[abc]} 字符类

\subsubsection{引号与变量展开}
\begin{lstlisting}
echo "$HOME"
echo '$HOME'
\end{lstlisting}
双引号会展开变量 \\ 单引号原样输出

\subsubsection{命令替换}
\begin{lstlisting}
echo "files: $(ls | wc -l)"
\end{lstlisting}
\texttt{\$( ... )} 将子命令输出嵌入当前命令参数

\subsubsection{提权写文件}
\begin{lstlisting}
echo 3 | sudo tee /tmp/demo
\end{lstlisting}
重定向由 shell 执行 \\ 使用 \texttt{sudo} 时可借助 \texttt{tee}

\subsubsection{find:按名称与类型查找}
\begin{lstlisting}
find . -name "*.txt" -type f
\end{lstlisting}
在当前目录递归查找普通文件

\subsubsection{find:按时间过滤}
\begin{lstlisting}
find . -mtime -1 -type f
\end{lstlisting}
筛选最近一天内修改的文件

\subsection{Shell脚本}

\subsubsection{shebang 与可执行权限}
\begin{lstlisting}
#!/usr/bin/env bash
# 保存为 scripts/demo.sh
# 赋予可执行权限:chmod +x scripts/demo.sh
# 运行:./scripts/demo.sh foo bar

echo "script path: $0"
echo "args: $@"
echo "count: $#"
\end{lstlisting}
\texttt{\#!/usr/bin/env bash} 通过 \texttt{PATH} 查找解释器,增强可移植性;\texttt{\$0} 为脚本名,\texttt{\$@} 为全部参数,\texttt{\$#} 为参数个数

\subsubsection{参数校验与循环}
\begin{lstlisting}
#!/usr/bin/env bash
# 统计传入文件的行数;若无参数则给出用法并退出

if [ $# -lt 1 ]; then
  echo "Usage: $0 FILE..." >&2
  exit 1
fi

for f in "$@"; do
  if [ -f "$f" ]; then
    wc -l < "$f"
  else
    echo "skip: $f (not a file)"
  fi
done
\end{lstlisting}
用 \texttt{[ ]} 做条件判断,\texttt{-f} 测试普通文件;\texttt{"\$@"} 保留每个参数的整体性;\texttt{exit 1} 表示非零退出码用于指示错误



\subsection{编辑器(Vim)}

\subsubsection{多模式与保存退出}
\begin{lstlisting}
# 正常模式 <ESC>
# 插入模式 i / a / o / O
# 命令行模式 :
:w      " 保存
:q      " 退出
:wq     " 保存并退出
:q!     " 强制退出
\end{lstlisting}


\subsubsection{基本移动}
\begin{lstlisting}
h  j  k  l        " 左下上右
w  b  e           " 以单词为单位前进/后退/到词尾
0  $              " 行首/行尾
gg  G             " 文件开头/文件结尾
\end{lstlisting}
在正常模式下用 \texttt{hjkl} 移动 \\ \texttt{w/b/e} 按单词移动 \\ \texttt{0/\$} 行首/行尾 \\ \texttt{gg/G} 文件首/尾

\subsubsection{查找与定位}
\begin{lstlisting}
/pattern          " 向下搜索
?pattern          " 向上搜索
n / N             " 下一个 / 上一个匹配
* / #             " 以光标下单词为关键词向下/向上搜索
\end{lstlisting}
使用正斜杠/问号进行前后向搜索 \\ \texttt{n/N} 在匹配间跳转 \\ \texttt{* / \#} 以当前单词为关键词快速定位

\subsubsection{操作符}
\begin{lstlisting}
d{motion}         " 删除到 {motion}
c{motion}         " 改写到 {motion}
y{motion} / p     " 复制(yank)/ 粘贴
u / <C-r>         " 撤销 / 重做
.                 " 重复上一次修改
\end{lstlisting}
\texttt{dw} 删除到下一个词起点,\texttt{cw} 改写该范围 \\ \texttt{.} 可重复上一步修改

\subsubsection{文本对象}
\begin{lstlisting}
ci"   ci'   ci(   ci[   " 改写引号/括号内文本
da"   da'   da(   da[   " 删除“包含定界符”的一段
\end{lstlisting}
结合 \texttt{i}/\texttt{a} 与引号、括号等精准作用于文本,如 \texttt{ci"} 与 \texttt{da(}

\subsubsection{可视模式}
\begin{lstlisting}
v     V     <C-v>       " 字符 / 行 / 块可视
y / d / > / <            " 复制 / 删除 / 增加缩进 / 减少缩进
gu / gU                  " 转小写 / 转大写
\end{lstlisting}
进入可视模式选择区域后,可进行复制、删除、缩进与大小写转换等操作

\subsubsection{缩进与格式化}
\begin{lstlisting}
>> / <<                " 行级缩进 / 反缩进
=                      " 对选中区域自动缩进
gg=G                   " 对全文进行缩进格式化
\end{lstlisting}

\subsubsection{缓冲区 / 窗口 / 标签页}
\begin{lstlisting}
:ls                    " 查看缓冲区
:b 2                   " 切到编号 2 的缓冲区
:split  /  :vsplit     " 水平 / 垂直分屏
:tabnew                " 新建标签页
:bd                    " 关闭当前缓冲区
\end{lstlisting}
通过缓冲区在内存中同时打开多个文件,结合分屏与标签页组织多文件编辑

\subsubsection{宏}
\begin{lstlisting}
q a        " 开始录制到寄存器 a
...        " 执行一系列编辑命令
q          " 结束录制
@a         " 回放宏 a
@@         " 重复上一次回放
\end{lstlisting}
把重复性编辑操作录成宏,批量回放以提升效率

\subsubsection{\texttt{\~/.vimrc}}
\begin{lstlisting}
" ~/.vimrc
set number
set ignorecase smartcase
set tabstop=2 shiftwidth=2 expandtab
set hlsearch incsearch
syntax on
filetype plugin indent on
\end{lstlisting}
开启行号、智能大小写搜索、统一缩进与语法高亮等基础配置

\newpage
\section{解题感悟}

通过本次实验,我体会到 Shell 在类 Unix(如 Linux)环境下能把零散的任务高效地“拼起来”借助重定向与管道把简单命令按需组合(例如 \texttt{cmd1 | cmd2 | cmd3}),再配合 \texttt{find}/\texttt{grep}(或 \texttt{rg})等基础工具,就能快速完成查找、过滤、统计等常见工作;把重复流程写成脚本并加入可读的参数与退出码,不仅省时,还提升了可重复性与可维护性。

\vspace{14pt}

Vim 则是在终端环境下非常顺手的文本编辑器:“动词+动作”让移动与修改形成可预测、可组合的操作;文本对象与可视模式能够精确地选中并批量改写结构化内容;\texttt{\~/.vimrc} 配置即可显著改善默认体验,而丰富的插件生态也支持按需扩展(如文件跳转、代码搜索等)。






\newpage

\begin{thebibliography}{9}
\bibitem{course-shell} Missing Semester 中文版:课程概览与 shell,\url{https://missing-semester-cn.github.io/2020/course-shell/}
\bibitem{shell-tools} Missing Semester 中文版:Shell 工具和脚本,\url{https://missing-semester-cn.github.io/2020/shell-tools/}
\bibitem{editors} Missing Semester 中文版:编辑器(Vim),\url{https://missing-semester-cn.github.io/2020/editors/}
\end{thebibliography}


\section*{GitHub 链接}
\begin{center}
\href{https://github.com/Misasasasasaka/report}{https://github.com/Misasasasasaka/report/tree/main/P2}
\end{center}

\end{document}
